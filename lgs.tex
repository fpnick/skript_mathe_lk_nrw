\[ a_1 x_1 + a_2 x_2 + a_3 x_3 + ... + a_n x_n = b \] hei�t \emph{lineare Gleichung} \\
\begin{tabular} {ll}
$x_i$ & Variablen \\
$a_i$ & Koeffizienten \\
$b$ & absolut gleich
\end{tabular}\\
$x_i$ sind stets linear (1. Potenz)

\section{Allgemeines Gleichungssystem}
\[ a_{11} x_1 + a_{12} x_2 + \ldots + a_{1n} x_n = b_1 \]
\[ a_{21} x_1 + a_{22} x_2 + \ldots + a_{2n} x_n = b_2 \]
\[ \vdots \]
\[ a_{m1} x_1 + a_{m2} x_2 + \ldots + a_{mn} x_n = b_n \]
$n$ Unbekannte \\
$m$ Gleichungen \\
$a_{ij}$ hei�en \emph{Koeffizienten}

\section{Koeffizientenmatrix}
\[ \left( \begin{array}{ccccc|c}
a_{11} & a_{22} & a_{33} & \ldots & a_{1n} & b_1 \\
a_{21} & a_{22} & a_{23} & \ldots & a_{2n} & b_2 \\
			 &				&				 & \vdots & 			 & 	\vdots \\
a_{m1} & a_{m2} & a_{m3} & \ldots & a_{mn} & b_n
\end{array}\right) \]
wobei $a_{1j}$ die Koeffizienten der Unbekannten in der ersten Gleichung sind, $a_{2j}$ die Koeffizienten in der zweiten Gleichung usw.

\section{Gau�-Verfahren}
Das Gau�-Verfahren dient zur �berf�hrung einer Matrix in Zeilenstufenform, sodass das zugeh�rige LGS einfach zu l�sen \begin{bsp}
\begin{align*}
\left( \begin{array}{ccc|c}
1 & 2 & 3 & 7 \\
0 & 4 & 0 & 6 \\
0 & -4 & -3 & -7 \\
\end{array}\right) \\
\left( \begin{array}{ccc|c}
1 & 2 & 3 & 7 \\
0 & 4 & 0 & 6 \\
0 & 0 & -3 & -1 \\
\end{array}\right) \\
\end{align*}
Diese Matrix ist in Zeilenstufenform, da sich unterhalb der Hauptdiagonalen (1, 4, -3) nur Eintr�ge = 0 befinden. Das Gleichungssystem ist nun einfach zu l�sen. \\
\end{bsp}

Erlaubte Umformungen im Gau�-Verfahren: \\
\begin{itemize}
	\item Zeilen komplett vertauschen. ($\stackrel{I \leftrightarrow III}{\longrightarrow}$)
	\item Multiplikation einer Gleichung mit einer Zahl verschieden von Null. ($\stackrel{I \cdot \lambda}{\longrightarrow}$)
	\item Addition von zwei Gleichungen. ($\stackrel{I + III}{\longrightarrow}$)
\end{itemize}

\subsection{Verschiedene F�lle}
An drei Beispielen:

\begin{bsp}
$\left( \begin{array}{cc|c}
2 & -1 & 1 \\
0 & 4 & 1
\end{array} \right)$
eindeutig l�sbar
\end{bsp}

\begin{bsp}
$\left( \begin{array}{ccc|c}
1 & -2 & -4 & 2 \\
0 & 3 & 2 & 0 \\
0 & 0 & 0 & -1
\end{array} \right)$
nicht l�sbar ($-1 \neq 0$)
\end{bsp}

\begin{bsp}
$\left( \begin{array}{ccc|c}
2 & -1 & 1 & 2 \\
0 & 2 & -6 & 0 \\
0 & 0 & 0 & 0
\end{array} \right)$\\
unendlich viele L�sungen (zwei linear abh�ngige Zeilen)\\
\begin{align*}
\mathbb{L} = \{ (1 - r | 3r | r)|r \in \mathbb{R} \}
\end{align*}
wobei $r$ \emph{L�sungsparameter} genannt wird.
\end{bsp}

\section{LGS mit Parameter (Schar von LGS)}
\[
\begin{array}{l}
	x_1 + x_2 - 2x_3 = 0 \\
	2x_1 - 2x_2 + 3x_3 = 1 + 2t \\
	x_1 - x_2 - x_3 = t
\end{array}
\]
$t$ hei�t \emph{Scharparameter} $\rightarrow$ unendlich viele lineare Gleichungssysteme\\
\textbf{Wichtig:} Beim L�sen eines solchen Systems m�ssen unter Umst�nden Fallunterscheidungen gemacht werden! (Nicht durch Null teilen!)

\section{Erweitertes Gau�-Verfahren}
Beim erweiterten Gau�-Verfahren versuchen wir, die Matrix nicht nur in Zeilenstufenform zu bringen, sondern sie zur \textit{Einheitsmatrix} umzuformen. \\
Wir m�ssen die Matrix also so umformen, dass alle Eintr�ge Null sind, au�er diejenigen auf der Diagonalen und diese sind gleich 1.\\

Beispiel f�r ein 3x3 System nach dem erweiterten Gau�-Verfahren:
\begin{bsp}
$\left( \begin{array}{ccc|c}
1 & 0 & 0 & b_1 \\
0 & 1 & 0 & b_2 \\
0 & 0 & 1 & b_3
\end{array} \right)$\\

Die L�sungen lassen sich nun ganz einfach ablesen:\\
$x_1 = b_1$ \\
$x_2 = b_2$ \\
$x_3 = b_3$
\end{bsp}