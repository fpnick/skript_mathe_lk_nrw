\textbf{�ber dieses Dokument:}\\
Dieses Dokument beruht auf den Mitschriften aus dem Mathematikunterricht von Frau M. Peters im Leistungskurs des Abiturjahrgangs 2008 am Nicolaus Cusanus Gymnasium Bergisch Gladbach (NRW). Seit dem Abitur im April 2008 wird dieses Dokument st�ndig Verbesserungen und Erweiterungen unterzogen.\\
Dies Aufzeichnungen sind weder vollst�ndig, noch vollst�ndig korrigiert. Es k�nnten also sowohl Rechtschreibfehler als auch \textbf{inhaltliche Fehler} vorhanden sein. Verbesserungsvorschl�ge, Korrekturen etc sind gerne willkommen unter fp.nick@gmail.com\\
\quad\\
Das Themengebiet \textit{Stochastik} fehlt in diesem Skript (noch), da es auch im Unterricht nur angerissen wurde und die Informationen somit recht beschr�nkt sind. Wer Interesse daran hat, zur Vervollst�ndigung dieses Skripts beizutragen, kann sich gerne �ber die unten angegebenen Kontaktdaten bei mir melden. LaTeX Kenntnisse sind dabei in jedem Fall hilfreich!\\
Weiterhin ist auch ein Teil mit �bungsaufgaben zu den einzelnen Kapiteln geplant. Wer hier mitwirken will ist ebenfalls herzlich willkommen!\\
\quad\\
Diskutiert werden kann unter \\
http://www.matheboard.de/thread.php?threadid=366144\\
\quad\\
Kontaktaufnahme zum Autor bitte unter fp.nick@gmail.com\\
\quad\\
Fabian Nick im Februar 2009
\newpage
\begin{center}
\textbf{Symbole}
\end{center}
\begin{center}
\begin{tabular}{|l|l|p{4cm}|}
\textbf{Symbol} & \textbf{sprachliche Fassung} & \textbf{Beispiel} \\
\hline
$\land$ & und & \\
$\lor$ & oder & \\
$\in$ & ist Element in / liegt in & $P \in g$ kann z.B. bedeuten, dass der Punkt $P$ auf der Geraden $g$ liegt.\\
$\cap$ & geschnitten mit & $E \cap g$ ist zum Beispiel der Schnitt einer Ebene $E$ mit der Geraden $g$, also in der Regel ein Punkt, oder aber auch die Gerade $g$ selbst.\\
$\subset$ & ist enthalten in & $g \subset E$ bedeutet, dass die Gerade $g$ vollst�ndig in der Ebene $E$ liegt.
$\|$ & parallel & \\
$\equiv$ & identisch & \\
$\bot$ & senkrecht (orthogonal) & \\
\hline
\end{tabular}
\end{center}